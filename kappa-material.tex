% Created 2015-11-17 Tue 21:05
\documentclass[11pt]{article}
\usepackage[utf8]{inputenc}
\usepackage[T1]{fontenc}
\usepackage{fixltx2e}
\usepackage{graphicx}
\usepackage{grffile}
\usepackage{longtable}
\usepackage{wrapfig}
\usepackage{rotating}
\usepackage[normalem]{ulem}
\usepackage{amsmath}
\usepackage{textcomp}
\usepackage{amssymb}
\usepackage{capt-of}
\usepackage{hyperref}
\author{William Henney}
\date{\today}
\title{Stuff for Gary's anti-Kappa paper}
\hypersetup{
 pdfauthor={William Henney},
 pdftitle={Stuff for Gary's anti-Kappa paper},
 pdfkeywords={},
 pdfsubject={},
 pdfcreator={Emacs 24.5.7 (Org mode 8.3.2)}, 
 pdflang={English}}
\begin{document}

\maketitle
\tableofcontents

\section{Important points for kappa paper}
\label{sec:orgheadline1}
\begin{itemize}
\item Deviations from a Maxwellian electron velocity distribution in a plasma arise when significant non-local transport of electrons occurs, which is important if there are steep gradients in physical conditions (such as T).
\item The dimensionless Knudsen number, Kn = \(\lambda_{\text{e }}\)/ L, which describes the "collisionality" of the plasma, is the most important parameter in determining the importance of these deviations.  This is the ratio between the electron elastic collisional mean free path, \(\lambda_{\text{e}}\), and the relevant length scale, L, which is the distance over which physical conditions change appreciably (for instance, the scale length of the temperature gradient: [d ln T / d s]\(^{\text{-1}}\))
\begin{itemize}
\item If Kn \(\ge\) 1, then the plasma is "non-collisional" and the electron velocity distribution will be very different from a Maxwellian, and also non-isotropic in the presence of a significant magnetic field.  An extreme example is the terrestrial magnetosphere, where Kn \(\gg\) 1.
\item If Kn < 1, then the plasma is "collisional" and as Kn \(\to\) 0 the electron velocity distribution will tend towards a Maxwellian distribution at the local temperature.  However, the fact that \(\lambda_{\text{e}}\) increases with electron velocity means that a significant high-energy non-Maxwellian tail can persist for rather small values of Kn.  Quantities that are sensitive to this tail, such as thermal conductivity or the collisional excitation of optical/UV emission lines, can significantly deviate from the Maxwellian values for Kn as low as 0.001
\item In H II regions, Kn \(\approx\) 1e-9 for the region as a whole, but certain sub-structures can have smaller values
\begin{itemize}
\item Ionization fronts (where hydrogen rapidly changes from being prefominantly ionized to predominantly neutral) have Kn \(\simeq\) 1e-6, but the emission from the ionization front is typically a tiny fraction of the total emission from the region (unless the ionization parameter is very low).
\item Cooling zones behind moderate velocity (20 - 100 km/s) shocks also have Kn \(\simeq\) 1e-6
\end{itemize}
\end{itemize}
\end{itemize}
\section{Question of magnetic fields}
\label{sec:orgheadline2}
\begin{itemize}
\item Although the typical \(\beta\) values for H II regions are > 1 (thermal pressure dominates magnetic pressure), that does not preclude the possibility of low-\(\beta\) regions of the nebula, where magnetic pressure dominates
\item In fact, this is suggested by simulations (Henney et al 2009; Arthur et al 2011)
\item \textbf{However} these will be in approximate balance of total pressure: P\(_{\text{M }}\)+ P\(_{\text{gas}}\)= P\(_{\text{gas}}\) (1 + 1/\(\beta\))
\item So, imagine we have a fraction x of the nebular volume being gas-dominated with \(\beta_{\text{1}}\) = 100, while a fraction (1 - x) is magnetically dominated with \(\beta_{\text{2}}\) = 0.01
\begin{itemize}
\item If T\(_{\text{2}}\) \(\approx\) T\(_{\text{1}}\), then n\(_{\text{2}}\) / n\(_{\text{1}}\) \(\approx\) P\(_{\text{2}}\) / P\(_{\text{1}}\)  = (1 + \(\beta_{\text{1}}^{\text{-1}}\)) / (1 + \(\beta_{\text{2}}^{\text{-1}}\)) = (\(\beta_{\text{2}}\) / \(\beta_{\text{1}}\)) (\(\beta_{\text{1}}\) + 1) / (\(\beta_{\text{2}}\) + 1) \(\simeq\) \(\beta_{\text{2}}\)
\item Volume Emission Measure: EM \(\propto\) V n\(^{\text{2}}\)
\item => EM\(_{\text{2}}\)/EM\(_{\text{1}}\) = (1 - x) \(\beta_{\text{2}}^{\text{2}}\) / x \(\simeq\) \(\beta_{\text{2}}^{\text{2}}\) / x if x is small
\item So if x = 0.01 (typical filling factor), and \(\beta_{\text{2}}\) = 0.01, then EM\(_{\text{2}}\)/EM\(_{\text{1}}\) = 0.01
\end{itemize}
\item \emph{This implies that the magnetically dominated gas contributes negligibly to the emission, even if it fills 99\% of the volume of the nebula!}
\end{itemize}
\section{General points about filling factor of H II region}
\label{sec:orgheadline3}
\begin{itemize}
\item Filling factor can come from 3 things:
\begin{enumerate}
\item Density structure within the "normal" ionized gas
\begin{itemize}
\item ifront closer to star in some directions than others: n\(^{\text{2}}\) h \(\propto\) Q / R\(^{\text{2}}\) (caused by inhomogeneities in neutral/molecular gas)
\item density fall along ionized photoevaporation flow n \textasciitilde{} 1 / v r\(^{\text{2}}\) or Bernoulli: ln n + 1/2 v\(^{\text{2}}\) = constant
\item low-velocity shocks (15-100 km/s), either
\begin{itemize}
\item caused by geometry readjustments (diverging flows on small scales turn into converging flows on larger scales - hello neighbor!)
\item jets from T Tauri stars, etc
\end{itemize}
\end{itemize}
\item Magnetically dominated regions (see \hyperref[sec:orgheadline2]{previous})
\item Hot gas from shocked winds
\begin{itemize}
\item similar arguments as for the low-\(\beta\) case, but with T\(_{\text{2}}\)/T\(_{\text{1}}\) > 100 instead of 1/\(\beta_{\text{2}}\)
\item so EM will be small, even if volume fraction is large
\item and additionally, emission spectrum will be X-rays rather than optical
\end{itemize}
\end{enumerate}
\end{itemize}

\section{Thermalization without collisions}
\label{sec:orgheadline4}
\begin{itemize}
\item The kappa hypothesis is that the electron velocity distribution is significantly non Maxwellian, despite the fact that all the indications are that the plasma is strongly collisional.

\item However, it is more often the case that the opposite is seen. Plasmas can be "thermalized", even if they are non collisional. This is what happens in shocks for instance, and is also what is described in Coulette \& Manfredi (2015). In their case, they say it is due to a velocity bunching like effect.

\item \textit{[2015-11-16 Mon 18:13] } Also, Laming (2004) suggests that collisionless lower hybrid waves can cause equilibration of the Te and Ti in the lower corona.
\end{itemize}
\section{{\bfseries\sffamily DONE} Message sent to Gary \textit{[2015-11-14 Sat]}}
\label{sec:orgheadline5}
I've been thinking about the kappa paper recently, on and off. I've had some ideas about how to frame it in a positive and constructive way, so that we will have no difficulty in publishing it as a research paper.    The idea would be to show exactly where in photoionized nebulae one should see non Maxwellian electrons.  For a given mechanism, for instance shocks, we can quantitatively estimate the relative contributions of "kappa" and "true" T structure to the apparent observed t\^{}2.  The "kappa" contribution will be shown to be negligible.

[Note that I haven't read a recent draft of your MS, so apologies if I am telling you things that you have already considered]

I think the key to this is the Knudsen number: Kn \textasciitilde{}  λ/L where λ is the collisional mean free path and L is the length scale of interest.  If Kn is less than one, then the plasma is said to be "collisional", whereas if it is of order 1 or greater then the plasma is "non-collisional".

\url{https://en.wikipedia.org/wiki/Knudsen_number}

All of the fields where kappa distributions are heavily used (solar wind, terrestrial magnetosphere, etc) are plasmas with Kn ≥ 1.  In H II regions, if we take L as the characteristic size of the object, then Kn = 1e-10 to 1e-8 over the whole range from proplyds up to the WIM.

So far, this argument pretty well mirrors your original discussion of timescales, but using length scales instead.  However, H II regions are not spatially homogeneous, and the advantage of discussing length scales is that we can easily accommodate that.

For instance, we see structure at the ionization front on scales of order the ionizing photon mean free path.  For Orion Huygens region, this is about 1e14 cm, giving Kn = 1e-6, so still strongly collisional at this scale.  We can go down even further to the Field length, which is about 1e11 cm in Orion (1e-5 arcsec, so not observable directly).  This is the scale at which heat conduction suppresses the growth of thermal instabilities. Even at this tiny scale, we have Kn = 1e-3, so the plasma is still collisional and deviations from Maxwellian will be small.

There is only one important scale that is smaller than the electron collisional λ, and that is the Larmor radius, which is about 1e4 cm in Orion.  So finally we have arrived at a scale on which the plasma can be considered non-collisional, with Kn = 1e4, so strong deviations from a Maxwellian will occur.  This is the gyroscopic radius of the helical motions of the electrons around the magnetic field lines.

This is important in determining the thickness of shocks in the ionized gas.  The shock itself will be non-collisional, mediated by self-generated MHD turbulence, and with thickness a few times the Larmor radius, so say 1e5 cm.  (The details depend on the angle between the magnetic field and the shock, but this does not matter much for our purposes.)

There will then be an electron thermalization layer of thickness a few times λ, so about 1e9 cm.  this is the region in which the kappa distribution will be most applicable.

After that, we have a non-equilibrium ionization layer, in which the ionization state of the gas adjusts to the  post shock temperature, followed by a cooling layer, in which the temperature will  decline from the post shock value back down to the photoionized equilibrium temperature.    The thickness of the ionization layer is about 1e11 cm and of the cooling layer from 1e12 to 1e14 cm, depending on the Mach number of the shock.  Therefore, their Knudsen numbers are 1e-6 to 1e-3, so the deviation from Maxwellian will be small, but not necessarily completely negligible.  I have ideas about how we could do a simplified Boltzmann equation model of these  regions, which allow us to predict the value of kappa.  Due to the elevated temperature, these are the regions that will contribute directly to t\^{}2.

Finally, we get the equilibrium shocked shell, which has roughly the same temperature as the H II region, but higher density.  The thickness of this depends strongly on the geometry and the shock Mach number, but values of 1e15 to 1e16 cm are typical, so Kn < 1e-7 and deviation from Maxwellian velocities should again be completely negligible.  This final layer will not contribute to the line-of-sight ADF t\^{}2, but it may contribute to the apparent plane-of-sky t\^{}2, since 2 or more different densities along the same line of sight can mimic a high T in the [N II] ratio.

Anyway, this message has got too long already, so I will stop now.  Let me know if you think any of this is worth pursuing.  (After I have finished the WFC3/MUSE analysis of course!). Comments from Bob and Manuel are welcome too

\section{\textit{[2015-11-15 Sun] } Material from Bradshaw \& Raymond (2013)}
\label{sec:orgheadline6}
\begin{itemize}
\item This is a really excellent article
\item Section 5.1 is the relevant one
\begin{itemize}
\item Discusses how to solve the Boltzmann equation and find the velocity distributions
\begin{itemize}
\item Starts with BGK approximation for the collisional term
\begin{itemize}
\item Improvements to take account of unequal electron and ion masses
\item And how to 
\begin{verbatim}
choose the correct parameters for the Maxwellians in the cross-collision terms to conserve density, momentum and energy.
\end{verbatim}
\item Greene, J. M. 1973, Physics of Fluids, 16, 2022
\end{itemize}
\item Then described Fokker-Planck approach
\begin{itemize}
\item Spitzer \& Härm (1953) was milestone
\begin{itemize}
\item Found modification to electron velocity distribution due to T and P gradients, and electric field E
\item Fractional change in f is of order \(\lambda\)/H where \(\lambda\) is the electron mean free path and H is the pressure or temperature scale height.  E.g., P/(d P/d z)
\item But multiplied by a factor that depends on particle speed v, and which can get large for v \(\gg\) v\(_{\text{thermal}}\)
\item So Spitzer \& Härm is only valid up to some critical velocity
\begin{itemize}
\item $\boxtimes$ Need to check what that is, once I get hold of the paper
\begin{itemize}
\item SH53 only consider velocities up to 3 times thermal
\item But the perturbative appproach breaks down for higher speeds
\end{itemize}
\end{itemize}
\end{itemize}
\item Extended by Ljepojevic, N. N., \& Burgess, A. 1990b, Proc. R. Soc. Lond. A, 428, 71
\begin{itemize}
\item Adds in treatment of high-velocity tail in approximation of neglecting self-interaction of high-velocity particles
\item $\boxtimes$ Need to read this - \textbf{another excellent paper}
\end{itemize}
\end{itemize}
\item Finally, mentions numerical solutions, e.g.
\begin{itemize}
\item Ljepojevic (1990)
\begin{itemize}
\item Photosphere to mid-transition region
\item Nearly Maxwellian
\end{itemize}
\item MacNeice et al (1991)
\begin{itemize}
\item Flaring loop
\item Enhanced tail populations
\end{itemize}
\end{itemize}
\end{itemize}
\end{itemize}
\item Section 5.2 has some interesting snippets too:
\begin{verbatim}
Shoub (1983) found significant deviations from Maxwellian in the tail of the distribution for Kn = 10−3,
\end{verbatim}
and
\begin{verbatim}
Owocki & Canfield (1986) used a BGK-type method to calculate the electron distribution function in the solar transition region to study the effect of a high-energy tail on the heat transport and collisional excitation and ionization rates.
\end{verbatim}
\end{itemize}
\section{Material from Dudik et al (2015)}
\label{sec:orgheadline10}
\subsection{More attempted observations of kappa in solar wind and corona}
\label{sec:orgheadline7}
\begin{itemize}
\item Solar wind, in  situ : \(\kappa\) \(\ge\) 2.5
\begin{itemize}
\item (Collier et al. 1996; Maksimovic et al. 1997a,b; Zouganelis 2008; Le Chat et al. 2011).
\end{itemize}
\item Si III spectra of transition region: κ ≈ 7
\begin{itemize}
\item (Dzifčáková \& Kulinová 2011)
\end{itemize}
\end{itemize}
\subsection{Mechanisms for producing \(\kappa\) distributions}
\label{sec:orgheadline8}
\begin{itemize}
\item Quote from intro
\begin{verbatim}
However, [The assumption of Maxwellian distribution] is incorrect if there are correlations between the particles in the system. Such correlations can be induced by any long-range interactions in the system
\end{verbatim}
\begin{itemize}
\item (Collier 2004; Leubner 2004; Livadiotis \& McComas 2009, 2010, 2013)
\end{itemize}
\item Examples
\begin{itemize}
\item particle acceleration due to magnetic reconnection
\begin{itemize}
\item (e.g., Zharkova et al. 2011; Petkaki \& MacKinnon 2011; Stanier et al. 2012; Cargill et al. 2012; Burge et al. 2012, 2014; Gordovskyy et al. 2013, 2014)
\end{itemize}
\item shocks, or wave- particle interactions
\begin{itemize}
\item (e.g., Vocks et al. 2008)
\end{itemize}
\end{itemize}
\item Carrying on
\begin{verbatim}
In such cases, the particle distribution will depart from the Maxwellian one, and will likely exhibit an enhanced high-energy tail. Furthermore, turbulence with the dif- fusion coefficient inversely proportional to particle ve- locity will also lead to the appearance of the non- Maxwellian distributions with characteristic high-energy tails
\end{verbatim}
\begin{itemize}
\item (e.g., Hasegawa et al. 1985; Laming \& Lepri 2007; Bian et al. 2014).
\end{itemize}
\end{itemize}
\subsection{Results of coronal loop ne, Te, \(\kappa\) diagnostics}
\label{sec:orgheadline9}
\begin{itemize}
\item Width of coronal loop is about 3 arcsec
\begin{itemize}
\item Radius of sun is 900 arcsec
\item So, about 0.0033 Rsun
\end{itemize}
\end{itemize}

\begin{center}
\begin{tabular}{lrrrrrrr}
Region & T / K & n / pcc & H & ln \(\Lambda\) & \(\lambda_{\text{e}}\) & K\(_{\text{n}}\) & \(\kappa\)\\
\hline
average loop & 3.2\,(6) & 1.8\,(9) & 3.33\,(-3) & 21.26 & 7.02\,(7) & 0.30 & 2\\
y=300-309 & 3.2\,(6) & 1.6\,(9) & 3.33\,(-3) & 21.32 & 7.88\,(7) & 0.34 & 2\\
\end{tabular}
\end{center}

\begin{itemize}
\item So \(\kappa\) is very low (2), but the Knudsen number is relatively large, although not that large
\item Also, we haven't included any radial T gradients
\begin{itemize}
\item If they are on smaller scale than 2e8 cm then they will affect Kn
\end{itemize}
\item And we haven't taken into account time-dependence
\item The microflares evolve on a timescale of minutes = 60 s
\begin{itemize}
\item electron speed is ve = sqrt(k T/m) = 7e8 cm/s = 7000 km/s
\item so electron collision time is 7e7 / ve = 0.33 s
\item so collision time / evolution time = 5e-3, which is smaller than Kn
\item \textbf{Conclusion:} It is steep spatial gradients rather than fast timescales that produce the non-Maxwellian distributions
\end{itemize}
\end{itemize}

\section{Material from Dzifcakova \& Kulinova (2011)}
\label{sec:orgheadline12}
\begin{itemize}
\item Diagnostics of the \(\kappa\)-distribution using Si III lines in the solar transition region
\item Scale heights we can calculate from hydrostatic equilibrium:
\begin{itemize}
\item H = c\^{}2 / g
\item g = G M / R\^{}2 = 6.673e-8 1.989e33 / 6.96e10**2 = 2.74e4
\item \(\rho\) c\^{}2 = 2 n k T => c\^{}2 = 2 k T / m
\item => H = 2 k T / m g
\end{itemize}
\item But these are far too large!
\begin{itemize}
\item The important thing is the T gradient (increasing outward), not the pressure gradient (decreasing outward)
\item From Table 3 of Shoub (1983), for n0 T0 = 6e14 K/cm3, we get this:
\begin{center}
\begin{tabular}{rrrrrrr}
z & T & n & H\(_{\text{T}}\) & ln \(\Lambda\) & \(\lambda_{\text{e}}\) & K\(_{\text{n}}\)\\
\hline
0 & 8.1\,(3) & 7.4\,(10) & 4.2\,(2) & 10.44 & 2.23\,(1) & 0.05\\
2.1\,(2) & 1.1\,(4) & 5.5\,(10) & 5.9\,(2) & 11.05 & 5.23\,(1) & 0.09\\
1.1\,(3) & 2.0\,(4) & 3\,(10) & 3.2\,(3) & 12.24 & 2.86\,(2) & 0.09\\
4.6\,(3) & 3.2\,(4) & 1.9\,(10) & 1.5\,(4) & 13.18 & 1.07\,(3) & 0.07\\
1.6\,(4) & 4.6\,(4) & 1.3\,(10) & 5.3\,(4) & 13.91 & 3.07\,(3) & 0.06\\
4.6\,(4) & 6.3\,(4) & 9.5\,(9) & 1.5\,(5) & 14.54 & 7.54\,(3) & 0.05\\
1.7\,(5) & 9.3\,(4) & 6.5\,(9) & 5.9\,(5) & 15.31 & 2.28\,(4) & 0.04\\
\end{tabular}
\end{center}
\end{itemize}
\end{itemize}

\subsection{Results of transition region diagnostics for T, n, \(\kappa\)}
\label{sec:orgheadline11}

\begin{center}
\begin{tabular}{lrrrrrrr}
Region & T / K & n / pcc & H & ln \(\Lambda\) & \(\lambda_{\text{e}}\) & K\(_{\text{n}}\) & \(\kappa\)\\
\hline
Coronal Hole & 2.5\,(4) & 1.4\,(10) & 6\,(3) & 12.96 & 9.04\,(2) & 0.15 & 13\\
Quiet Sun & 3.5\,(4) & 1.8\,(9) & 1.5\,(4) & 14.49 & 1.23\,(4) & 0.82 & 10\\
Active Region & 1\,(4) & 1.3\,(10) & 5.9\,(2) & 11.62 & 1.74\,(2) & 0.29 & 7\\
\end{tabular}
\end{center}

Note that 
\section{Material from Ljepojevic \& Burgess (1990)}
\label{sec:orgheadline18}
\begin{itemize}
\item Extends Spitzer \& Härm (1953) to include high-velocity electrons in a strong T gradient
\end{itemize}
\subsection{LB90 Methodology}
\label{sec:orgheadline13}
\begin{itemize}
\item Velocity in thermal units is \(\xi\) \(\equiv\) (m v\(^{\text{2 }}\)/ 2 k T)\(^{\text{1/2}}\)
\item Collision mean free path increases with elctron velocity as \(\lambda\) \(\propto\) v\(^{\text{4}}\)
\item Divide electrons into two parts:
\begin{enumerate}
\item Bulk is a nearly-thermal core (\(\xi\) < \(\xi_{\text{c}}\)), treated by SH53 perturbation method
\item Plus a high-velocity tail, treated by a their "High-velocity Vlassov-Landau" (HVL) approximation (pretty complicated!)
\end{enumerate}
\item Solutions are matched at \(\xi_{\text{c}}\) = 2, where both approximations are valid.
\item They calculate results for a plane-parallel slab with a T gradient between two constant regions at T\(_{\text{1}}\) and T\(_{\text{2}}\)
\item Boundary conditions are Maxwellian velocities at the two temperatures as \(z \to \pm\infty\)
\item To conserve charge neutrality an electric field E builds up, which gives a return current of thermal particles to balance the current of HV particles that stream down the T gradient: \[E = -0.703 \frac{4\pi \epsilon_{0}k}{e} \, \frac{d T}{d z}\]
\item The equations are non-dimensionalized:
\begin{itemize}
\item \[\tau(z) = \int_{0}^{z} \frac{1}{\lambda(z')}  d z'\]is like a "collisional depth".  Note the obvious analogy with radiative transfer here: 1/\(\lambda\) is an absorption coefficient.  It gets lower as the T gets higher.   The difference with stellar atmospheres is that there is no vacuum boundary on the RHS.  Instead, we tend to thermalization on both sides.
\item Their quantity \[\alpha(\tau) = \lambda \frac{1}{T} \frac{dT}{dz}\] is basically the same as Kn
\item The distribution function f is transformed to \[\phi = \frac{v_\text{th}^{3}}{n_{e}} f_{e}\]
\end{itemize}
\item Then they do \emph{another} transformation to deal with the fact that \(\phi\) varies by many orders of magnitude:
\begin{itemize}
\item \[ \phi = \pi^{-3/2} C \exp(-\xi^{2} g)\]
\item or \[ g = -\xi^{-2} \ln(\pi^{3/2} \phi / C)\], where C is a constant determined from normalization condition
\end{itemize}
\end{itemize}
\subsection{LB90 Results}
\label{sec:orgheadline16}
\begin{itemize}
\item They use empirical T, n distributions for the transition region from McWhirter et al (1977) and Burton et al (1971)
\begin{itemize}
\item The lowest regions have T = 15,000 (McWhirter) - 25,000 (Burton )K, n \(\approx\) 1e10 pcc and \(\alpha\) of order 1e-4 (Burton) to 1e-3 (McWhirter)
\item In the McWhirter data, \(\alpha\) is roughly constant at 1e-3 from 15,000 - 50,000 K (\(\tau\) = 0 \(\to\) 1000), then increases gradually to 3.5e-3 from 50,000 to 800,000 K (\(\tau\) = 1000 \(\to\) 2500), then falls quickly to 4e-4 from 800,000 to 1.2e6 K (\(\tau\) = 2500 \(\to\) 2600), as the T profile levels off.  So, in all positions the plasma is quite collisional for thermal speeds
\item In the Burton data, \(\alpha\) increases monotonically with height from 2e-4 at 24,000 K, through 2.5e-3 at 50,000 K (\(\tau\) = 1000), then 2e-2 at 100,000 K (\(\tau\) = 1170), then 6e-2 at 200,000 K (\(\tau\) = 1191), up to 0.1 at 300,000 K (\(\tau\) = 1198).  The T profile never turns over in this data.
\item For our purpose, we are really only interested in the velocity distributions in the lower part of the T ramp, where we expect fat tails from the hotter electrons coming down the gradient
\end{itemize}
\item They calculate what they call the "isotropic part of the normalized distribution function", which is akin to the mean intensity in radiative transfer: \[\phi_{_{0}} = \frac12 \int_{0}^{\pi} \phi \sin\theta\, d\theta\]
\item Then they also show results as function of \(\theta\)
\item $\boxtimes$ Tables of results are given below
\begin{itemize}
\item They are plotted
\end{itemize}
\end{itemize}
\subsubsection{LB90 Table from McWhirter data}
\label{sec:orgheadline14}
\begin{itemize}
\item Results for \(\phi_{\text{0}}\)/\(\phi_{\text{M}}\) from Table 4, incorporating Kn, or \(\alpha\), from Table 2
\end{itemize}
\begin{center}
\label{tab:orgtable1}

\begin{tabular}{rrrrrrrrl}
 & 2.5\,(4) & 3.2\,(4) & 6.4\,(4) & 1.28\,(5) & 2.56\,(5) & 5.12\,(5) & 1.17\,(6) & <- T\\
\(\xi\) & 1\,(-3) & 1.05\,(-3) & 1.3\,(-3) & 2.1\,(-3) & 2.6\,(-3) & 3.2\,(-3) & 3.8\,(-4) & <- Kn\\
\hline
2.5 & 0.99 & 1.0 & 1.0 & 1.0 & 1.0 & 1.0 & 0.99 & \\
3 & 0.99 & 0.99 & 0.99 & 1.0 & 1.0 & 1.0 & 0.97 & \\
3.5 & 0.99 & 1.0 & 1.01 & 1.03 & 1.06 & 1.09 & 0.93 & \\
4 & 1.03 & 1.04 & 1.10 & 1.24 & 1.38 & 1.59 & 0.86 & \\
4.5 & 1.20 & 1.22 & 1.49 & 2.20 & 3.03 & 4.70 & 0.78 & \\
5 & 1.89 & 1.96 & 3.39 & 9.01 & 20.0 & 39.0 & 0.71 & \\
5.5 & 4.96 & 4.96 & 25.6 & 1.60\,(2) & 5.56\,(2) & 7.99\,(2) & 0.68 & \\
6 & 32.3 & 43.7 & 1.36\,(3) & 1.21\,(4) & 4.39\,(4) & 2.84\,(4) & 0.68 & \\
\end{tabular}
\end{center}
\subsubsection{LB90 Table from Barlow data}
\label{sec:orgheadline15}
\begin{itemize}
\item Results for \(\phi_{\text{0}}\)/\(\phi_{\text{M}}\) from Table 4, incorporating Kn, or \(\alpha\), from Table 3
\end{itemize}
\begin{center}
\label{tab:orgtable2}

\begin{tabular}{rrrrrrl}
 & 2.5\,(4) & 3.2\,(4) & 6.4\,(4) & 1.28\,(5) & 2.56\,(5) & <- T\\
\(\xi\) & 2.2\,(-4) & 6.2\,(-4) & 5.9\,(-3) & 2.8\,(-2) & 7.9\,(-2) & <- Kn\\
\hline
2.5 & 1.0 & 1.01 & 1.03 & 1.16 & 1.17 & \\
3 & 1.0 & 1.03 & 1.23 & 2.37 & 2.30 & \\
3.5 & 1.0 & 1.12 & 2.92 & 12.4 & 8.97 & \\
4 & 1.01 & 1.50 & 32.3 & 1.50\,(2) & 67.7 & \\
4.5 & 1.01 & 12.7 & 1.18\,(3) & 3.31\,(3) & 9.64\,(2) & \\
5 & 1.02 & 1.15\,(3) & 1.02\,(5) & 1.55\,(5) & 2.66\,(4) & \\
5.5 & 3.64 & 1.37\,(5) & 8.07\,(6) & 6.38\,(6) & 9.37\,(5) & \\
6 & 85.3 & 1.66\,(7) & 3.34\,(9) & 1.79\,(9) & 1.40\,(8) & \\
\end{tabular}
\end{center}
\subsection{LB90 Discussion}
\label{sec:orgheadline17}
\begin{itemize}
\item Departures of \(\phi_{\text{0}}\) from Maxwellian occur for \(\xi\) > 3, so (E / kT) > 9
\begin{itemize}
\item Similar to \(\kappa\) distributions for \(\kappa\) > 10
\end{itemize}
\item Backscattering of downward moving electrons is the main source of upward moving electrons in the high-velocity tail
\item Turbulence was neglected.  This would increase collision frequency and decrease the deviations from Maxwellian.
\begin{itemize}
\item Ion-acoustic turbulence in presence of strong dT/dz was studied by Gray \& Kilkenny (1980)
\item Important for Kn > 0.4, above the values considered in this paper
\end{itemize}
\end{itemize}

\section{Make a graph of \(\kappa\) versus Kn}
\label{sec:orgheadline24}


\begin{itemize}
\item This would use some of the papers cited in the Bradshaw \& Raymond review
\item \(\kappa\) = 2.5 in velocity filtration models of coronal heating
\begin{itemize}
\item Anderson, S.W., Raymond, J.C. \& van Ballegooijen, A. 1996, ApJ, 457, 939
\end{itemize}
\item Base of corona, up through solar wind acceleration site, up to a few solar radii
\begin{itemize}
\item Maxwellian at base, but very non-Maxwellian at few solar radii
\item Esser, R., \& Edgar, R. J. 2000, ApJ, 532, 71
\item This is important because will cover a range of Kn I hope
\item 
\end{itemize}
\end{itemize}
\subsection{{\bfseries\sffamily DONE} Esser \& Edgar analysis}
\label{sec:orgheadline19}
\begin{itemize}
\item I will calculate Kn for different radii in their model (Fig 1)
\item And will also estimate \(\kappa\) from their arguments about their Fig 2
\begin{itemize}
\item They have a halo/core T ratio and n ratio, which we will have to translate into a \(\kappa\)
\end{itemize}
\item This works great - see table!
\end{itemize}
\begin{center}
\begin{tabular}{rrrrrrrrrr}
R/Rsun & T / K & n / pcc & H & ln \(\Lambda\) & \(\lambda_{\text{e}}\) & K\(_{\text{n}}\) & n\(_{\text{h}}\)/n\(_{\text{c}}\) & T\(_{\text{h}}\)/T\(_{\text{c}}\) & \(\kappa\)\\
\hline
1.0 & 5\,(5) & 3.8\,(8) & 0.07 & 19.26 & 8.97\,(6) & 1.8\,(-3) & 0.05 & <2 & 20\\
1.25 & 9\,(5) & 1\,(7) & 0.07 & 21.96 & 9.68\,(8) & 0.20 & 0.05 & 5 & 3\\
1.5 & 9\,(5) & 1\,(6) & 0.2 & 23.11 & 9.20\,(9) & 0.66 &  &  & \\
2.0 & 7\,(5) & 2\,(5) & 0.4 & 23.54 & 2.73\,(10) & 0.98 &  &  & \\
2.4 & 6\,(5) & 1\,(5) & 0.4 & 23.65 & 4.00\,(10) & 1.44 & 0.2 & 18 & 2\\
\end{tabular}
\end{center}

\subsection{{\bfseries\sffamily DONE} Equivalences between \(\kappa\) and core/halo distros}
\label{sec:orgheadline23}
\subsubsection{Kappa}
\label{sec:orgheadline20}
\[
f_{\kappa}(E) = 
A_{\kappa} \frac{2}{\sqrt{\pi}}
\left(\frac{1}{k T}\right)^{3/2}
\frac{\sqrt{E}}{
\left(1 + \frac{E}{(\kappa - 3/2)\, k T}\right)^{\kappa + 1}
}
\] 
where
\[
A_{\kappa} = \frac{ \Gamma(\kappa + 1) }{\Gamma(\kappa - 0.5) \, (\kappa - 1.5)^{3/2}}
\]


\subsubsection{Core/halo}
\label{sec:orgheadline21}
Single Maxwellian energy distribution per dE is 
\[
f_{M}(E) = \frac{2}{\sqrt{\pi}}
\left(\frac{1}{k T}\right)^{3/2} E^{1/2}\, e^{-E/k T}
\]

So a core-halo will be the sum of two of these.  Putting \(a = T_{H} / T_{C}\) and \(b = n_{H}/n_{C}\), we have
\[
f_{C-H}(E) = \frac{2}{\sqrt{\pi}}
\left(\frac{1}{k T}\right)^{3/2} (1 + b)^{-1} E^{1/2}\,
\left[e^{-E/kT} + (b / a^{3/2}) e^{-E/akT}\right]
\]
in which T is the core temperature

\subsubsection{Plot the distributions}
\label{sec:orgheadline22}
\begin{itemize}
\item Put kT = 1 and ditch the \((m / 2 \pi k T)^{3/2}\) term since it is the same for all
\item I am plotting ratio with maxwell, sonce that seems best
\item There are still a few problems
\begin{itemize}
\item I had to multiply the others by sqrt(E) to get them to look like the kappa ones
\item The kappa distros don't seem to integrate to the same value
\begin{itemize}
\item $\boxtimes$ how are they normalized? \emph{fixed now} \textit{[2015-11-16 Mon 19:38]}
\end{itemize}
\end{itemize}
\item Now to compare the core-halo to the kappa
\begin{itemize}
\item I am looking around E = 10 k T
\item Seems that the TH/TC = 2 curve is close to \(\kappa\) = 20
\item TH/TC = 5 => \(\kappa\) = 3
\item TH/TC = 18 => \(\kappa\) \(\sim\) 2
\item All these are approximate since the core halo distros are closely Maxwellian for E < 5 kT, wheras \(\kappa\) start deviating at about 3 k T
\end{itemize}
\end{itemize}
\begin{verbatim}
from __future__ import print_function
import sys
from matplotlib import pyplot as plt
import seaborn as sns
from scipy.special import gamma
import numpy as np
from numpy import exp, sqrt

def A_kappa(kappa):
    return gamma(kappa+1)/gamma(kappa-0.5)/(kappa-1.5)**1.5


def f_M(E):
    return sqrt(E) * exp(-E)


def f_CH(E, a, b):
    return sqrt(E) * (exp(-E) + (b/a**1.5)*exp(-E/a))/(1 + b)

def f_kappa(E, kappa):
    return A_kappa(kappa) * sqrt(E) / (1 + E/(kappa - 1.5))**(kappa + 1)


energy = np.logspace(-2, 2, 500)

fig, ax = plt.subplots(1, 1)
ax.plot(energy, 1e7*f_M(energy), lw=7, alpha=0.1, color='k', label='Maxwellian, $10^{7} f_M$')
for kappa in 1.75, 3.0, 5.0, 10.0, 20.0, 100.0:
    ax.plot(energy, f_kappa(energy, kappa)/f_M(energy), lw=3, alpha=0.5, label=r'$\kappa = {:.1f}$'.format(kappa))
for a, b in (2, 0.05), (5, 0.05), (18, 0.2):
    ax.plot(energy, f_CH(energy, a, b)/f_M(energy), ls='--', lw=1.5, label='$T_C/T_H = {}$; $n_C/n_H = {:.2f}$'.format(int(a), b))

ax.set_xscale('log')
ax.set_yscale('log')
ax.set_ylim(0.1, 3e7)
ax.legend(fontsize='small', loc='middle left', ncol=2)
ax.set_xlabel(r'$E\, /\, k T$')
ax.set_ylabel(r'Excess over Maxwellian: $f\, /\, f_M$')
figname = sys.argv[0].replace('.py', '.pdf')
fig.set_size_inches(7, 5)
fig.tight_layout()
fig.savefig(figname)
print(figname)
\end{verbatim}

\begin{verbatim}
python non-maxwell-distros.py
\end{verbatim}

\section{Calculate BGK model for photoionized equilibrium}
\label{sec:orgheadline28}

\subsection{Relaxation timescales}
\label{sec:orgheadline25}
\subsection{Note on Solving the Boltzmann equation}
\label{sec:orgheadline26}
For small deviations from Maxwellian we can use the Crook approximation to the elastic collisional terms. This is a simple relaxation term and saves having to solve the full Boltzman collision integral. This should be sufficient for calculating the effects of photoionization and recombination on the electron velocity distribution.

The simplest version has an interaction timescale τ that is independent of velocity, but extensions to τ increasing with v are simple I think.

We would be looking for steady state solutions to the Boltzmann equation. And to start with, ignoring the advection terms and the Lorentz force due to B field.

So it would just be
(df/dt)[ioniz] + (df/dt)[recomb] + (df/dt)[coll] = 0

The recomb rate (negative (df/dt)) is higher for lower velocities (sub thermal), while photoionization (positive (df/dt)) will produce super thermal electrons, particularly for hard ionizing spectrum. However, if we want to conserve energy and have a realistic T, we need to add in extra cooling processes. The simplest one would be a collisionally excited emission line, with a threshold energy ε > k T. This would give an extra term in the Boltzmann equation: (df/dt)[cool], which will have a negative and positive part. Negative for electron energies E > ε, to represent the electrons that excite the line, with a mirrored positive part for E - ε, to represent the post collision electrons.

\subsection{Where the Crook approach breaks down}
\label{sec:orgheadline27}
Suppose we start off with the sum of two Maxwellian distributions, and we let them evolve with time, under the influence of only elastic collisions. We have N1 particles with temperature T1 and N2 particles with temperature T2. The average temperature is T* = (N1 T1 + N2 T2) / (N1 + N2), which characterizes f\_M, which is the distribution we will relax towards.

Crook formula will work ok so long as there is substantial overlap between the three distributions. But in more extreme situations it will maybe fail.

For instance, consider N1=1e2 pcc, T1=1e4 K; N2= 1 pcc, T2=1e6 K, so that T*=2e4 K, more or less.

Around 1e5 K we will initially have f ≅ 0.1, or so from the low energy side of the second component. Whereas f\_M(T*) will be around 1e2 e\^{}-10 = 0.0045, which is 20 times smaller. So the crook collisions will make f fall with time here. But this actually seems reasonable.

The problem is that this will not conserve energy in the medium term. The low velocity electrons will quickly accommodate to the resolved temperature, but the relaxation time scale is much longer for the high velocity electrons, so the energy for the resolved Maxwellian is not available yet!

One solution would be to have a time-dependent resolved Maxwellian, which would have the energy of all the low-T component, plus that fraction of the high-T component with v < v’, where v’ is the velocity where the relaxation time is equal to A t, where t is the current time, and A is a constant of order unity.

This way, T* will evolve from 10,000 up to 20,000 K with time (in my example), as more and more of the high T component start to have collisions. The low velocity electrons will relax to the current f\_M( T* ) quicker than T* is changing, so they will just follow the evolving Maxwellian, whereas the highest velocity electrons will have an f that slowly drops down with time. Once v’ has got past the peak of the initial T2 distribution, then T* will have almost reached its final value, so the core of f will hold steady thereafter. Meanwhile, the high velocity remnant tail of still-uncollided electrons will be of higher and higher velocity, but lower and lower amplitude.

Next job: include cooling as well! The heating/cooling timescales (\textasciitilde{}1e10 s @ 100 pcc) are much slower than the collisional timescales at 10,000 K (200 s), but at 1,000,000 K the collisions are 1e5 times slower (τ goes as v\^{}5 at high energies). But this is still much smaller than the cooling timescale, so the thermalization takes about 1 year, producing a 2 times T increase, which is then radiated away over 300 years. (If we used 1e4 for the density instead of 100, then all the timescales would be 100 times shorter.) So, once again we find that the non Maxwellian effects are far less important than the T fluctuations that would ineluctably follow them.

On the other hand, if the initial high T component were at 1e7 K instead, then the timescales would be comparable, since the collision times would be 10\^{}5/2 = 300 times longer.

And if we took 1e8 K, as in a 2000 km/s shock, then we would have the opposite regime where the cooling is much faster than the collisions of the high velocity gas. In this case we can hold the "target" f\_M fixed at 10\^{}4 K, since cooling allows the T to remain constant while the high-velocity electrons are being thermalized.

But in this case, the total density of the high velocity component is only 1e-4 of the total, so f in the intermediate velocity range is hardly effected.
\section{Summary of conclusions about collisionality}
\label{sec:orgheadline29}
\begin{itemize}
\item H II regions are "non-collisional" plasmas in the sense that r\(_{\text{L}}\) \(\ll\) \(\lambda_{\text{e}}\)
\item But they are strongly "collisional" in the sense that r\(_{\text{L}}\) \(\ll\) R
\item It all depends on the scale that one is interested in.
\item See discussion in \hyperref[sec:orgheadline5]{letter sent to Gary}
\end{itemize}
\section{{\bfseries\sffamily TODO} Write up that table I did of the Knudsen number}
\label{sec:orgheadline30}
\begin{itemize}
\item Knudsen number K\(_{\text{n}}\) is the ratio between electron mean free path and size of region
\item Kappa distribution is used for the
\begin{description}
\item[{Solar wind}] K\(_{\text{n}}\) = 1 -- 10
\item[{Terrestial Magnetosphere}] K\(_{\text{n}}\) \(\simeq\) 10\(^{\text{8}}\)
\end{description}
\item H II regions have
\begin{description}
\item[{Galactic WIM}] K\(_{\text{n}}\) \(\simeq\) 4 \texttimes{} 10\(^{\text{-8}}\) (\(\lambda_{\text{e}}\) = 10\(^{\text{13}}\) cm)
\item[{Extended Orion Nebula}] K\(_{\text{n}}\) \(\simeq\) 4 \texttimes{} 10\(^{\text{-9}}\) (\(\lambda_{\text{e}}\) = 10\(^{\text{10}}\) cm)
\item[{Orion Nebula Core}] K\(_{\text{n}}\) \(\simeq\) 5 \texttimes{} 10\(^{\text{-10}}\) (\(\lambda_{\text{e}}\) = 10\(^{\text{8}}\) cm)
\item[{Proplyd}] K\(_{\text{n}}\) \(\simeq\) 5 \texttimes{} 10\(^{\text{-10}}\) (\(\lambda_{\text{e}}\) = 10\(^{\text{6}}\) cm)
\end{description}
\item Of course, if we look at a tiny region of the nebula, then the Knudsen number would be larger
\begin{itemize}
\item But there is no evidence for structure to the nebula on such tiny scales
\item And thermal conduction should smooth things out below 1e11 cm
\begin{itemize}
\item (Field length is proportional to 1/n, same as mean free path, so it is always 1000 times mean-free-path)
\end{itemize}
\end{itemize}
\item For all photoionized regions we have:
\begin{itemize}
\item \(\ell\) \(\ll\) \(\lambda_{\text{D}}\) \(\ll\) r\(_{\text{L}}\) \(\ll\) \(\lambda_{\text{e}}\) \(\ll\) l\(_{\text{f}}\) \(\ll\) \(\lambda_{\gamma}\) \(\ll\) R
\item New one here is \(\lambda_{\gamma}\) = 10 / n \(\sigma_{\text{0}}\) \(\simeq\) 2e14 cm for n = 1e4 and \(\sigma\)  = 6e-18
\end{itemize}
\item For Orion core this is
\begin{itemize}
\item 0.05 \(\ll\) 7 \(\ll\) 2.2e4 \(\ll\) \textbf{1.41e8} \(\ll\) 1e11 \(\ll\) 2e14 \(\ll\) 3e17 cm
\item The electron mean free path is highlighted in bold
\end{itemize}
\end{itemize}
\section{What about the Braginskii x parameter}
\label{sec:orgheadline31}
\begin{itemize}
\item This is the ratio of cyclotron frequency to collision frequency. E.g., x\(_{\text{e}}\) = \(\omega_{\text{c,e}}\) \(\tau_{\text{e}}\)
\item So it should be about the same as mean free path over Larmor radius
\begin{itemize}
\item Which is about x = 6400 for Orion
\end{itemize}
\end{itemize}
\section{Plasma parameter and plasma frequency}
\label{sec:orgheadline32}
\begin{itemize}
\item See Howard (2002), Introduction to Plasma Physics C17 Lecture Notes
\item \emph{Plasma parameter} \(\Lambda\) = n  \(\lambda_{\text{D}}^{\text{3}}\) is number of particles inside a Debye volume
\begin{itemize}
\item In principle, this is the same as in the ln \(\Lambda\) that we use in the mean free path calculation
\item Quote from Howard:
\begin{quote}
Λ is known as the plasma parameter. It is the only dimensionless parameter that characterises unmagnetized plasma systems. We idenitify two limits for Λ – the strongly coupled case Λ ≪ 1 in which the potential energy of the interacting particles is more significant than thier kinetic motions and the weakly coupled case Λ ≫ 1 where the particle thermal motions are more important. This is the case almost always encountered for naturally occurring and man-made plasmas.
\end{quote}
\item For H II regions this always very high: 10\(^{\text{7}}\) to 10\(^{\text{9}}\), being higher at lower densities
\end{itemize}
\item \emph{Plasma frequency} ω\(_{\text{p}}\) ∼ v\(_{\text{th }}\)/ λ\(_{\text{D}}\)
\begin{itemize}
\item For electrons, this is (k T / m λ\(_{\text{D}}^{\text{2}}\))\(^{\text{1/2}}\) = 5.6 MHz for n = 1e4 pcc, dropping as n\(^{\text{-1/2}}\)
\end{itemize}
\end{itemize}
\section{{\bfseries\sffamily NEXT} Use equations in Plasma Formulary}
\label{sec:orgheadline33}
\begin{itemize}
\item \includegraphics[width=.9\linewidth]{/Users/will/Dropbox/Documents/Ebooks/Wiley/Diver-PlasmaFormulary/ch6.pdf}
\item Section 6.3.1.3 looks relevant for Gary's kappa paper
\item Use equation 6.8 to find Maxwellian relaxation time as a function of particle energy
\end{itemize}
\section{{\bfseries\sffamily NEXT} Solve Boltzmann equation and estimate kappa}
\label{sec:orgheadline34}
\begin{itemize}
\item Use the Krook collision term, which is a good approzimation for interactions between like particles (e.g., electron-electron collisions)
\item Use equations from Howard (2002)
\end{itemize}
\section{What does it mean for a plasma to be "collisionless"}
\label{sec:orgheadline35}
\begin{itemize}
\item According to Wikipedia
\begin{verbatim}
In plasma physics of tokamaks, collisionality is a dimensionless parameter which expresses the ratio of the electron-ion collision frequency to the banana orbit frequency.
\end{verbatim}
\item In our case, we want to substitute cyclotron frequency for "banana orbit frequency", in which case this becomes similar to 1/x where x is the Braginskii parameter
\item Except that it talks about electron-ion collisions, whereas \(\tau_{\text{e}}\) is all about electron-electron I think
\item Neglecting that little detail, this implies that H II regions are still "collisionless" in this sense
\item So a shock transition can be mediated at scales of r\(_{\text{L }}\)but the post shock particles would not thermalise until \(\lambda_{\text{e}}\)
\item And all this is for electrons - ions will be different
\item But other authors compare with the size of the system L
\begin{itemize}
\item This is what Howard says:
\begin{quote}
The plasma “collisionality” often refers to a dimensionless measure such as ν/ω\(_{\text{T}}\) where ν is the actual collision frequency and ω\(_{\text{T}}\) is the system transit frequency. An alternative and more intuitive measure is the ratio 

λ\(_{\text{mfp}}\) / L ∼ ω\(_{\text{T}}\) / ν (1.18) 

where

λ\(_{\text{mfp}}\) ≡ v\(_{\text{th }}\)/ ν (1.19)

defines the mean free path between collisions. A “collisionless” plasma satisfies the condition λ\(_{\text{mfp}}\) >> L. 
\end{quote}
\item All this is reconciled in the \hyperref[sec:orgheadline5]{message sent to Gary}
\end{itemize}
\end{itemize}
\end{document}
